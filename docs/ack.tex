%%
% 致谢
% 谢辞应以简短的文字对课题研究与论文撰写过程中曾直接给予帮助的人员(例如指导教师、答疑教师及其他人员)表示对自己的谢意,这不仅是一种礼貌,也是对他人劳动的尊重,是治学者应当遵循的学术规范。内容限一页。
% modifier: 黄俊杰
% update date: 2017-04-15
%%

\chapter{致谢}

走在大学求学之路的四年有如白驹过隙,我的求学生涯就此告一段落。
回顾往昔,贝海拾遗,我深知这宝贵的四年不仅充实了计算机领域的知识技能,培育了理性唯物的科学素养,更重要的是,它锤炼了我深入探索无限进步的意志。
以此来看,在中山大学求学生活的宝贵经历,必然将成为我一生珍藏的财富。
在这一荣幸的也是幸福的时刻,我由衷地感谢所有支持与陪伴我的老师、家人和同学们。  

我感谢我的指导老师卢宇彤教授。
鉴于我是一名本科生,在毕业设计的相关工作中初次接触学术研究的冰山一角,不但缺乏经验,难以抓住研究问题的主干,也难以客观评判自身工作的价值。
卢老师凭借她对高性能计算领域的深刻理解,为我提供了宝贵的指引,使我在迷茫的泥潭中建设出一条可行的路途。
在办公室反复推敲最后得到的方案,在稿件上写满的一页页批注,使我领悟到学术的严谨,更让我领会师生情谊的温度。
卢老师所秉持的严谨治学态度和她那勤勉的工作精神,像永悬天北的吉星指引我的路途。
我谨向卢老师献上最崇高的敬意与最真挚的谢意。 

我也要感谢我的家人。
在求学路上的一道道难关面前,是亲人的陪伴与无条件的理解赋予我无穷的自信与力量,正是他们始终给予的无私鼓励与全力支持,让能够让我坚守信念、砥砺前行。
正是有了他们的付出与陪伴,我才能心无旁骛地追逐目标,在求学的道路上自信前进。
我向他们报以无穷的感激之情。

最后,我要感谢所有同窗挚友,为我们在图书馆共享过的晨昏,为我们在实验室见证过的星光。
无论北上广深,中山大学的学子彼此始终连着一根同学情谊的红线,共同的求学经历是我们情谊的基石,响亮的校训是不变的气概。
朋友们的热情支持与真诚理解,始终是我前行路上最坚实的依靠。
我将铭记初心,勇往直前。

\vskip 108pt
\begin{flushright}
	卢科州\makebox[1cm]{} \\
	\today
\end{flushright}

