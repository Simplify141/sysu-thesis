%%
% 致谢
% 谢辞应以简短的文字对课题研究与论文撰写过程中曾直接给予帮助的人员(例如指导教师、答疑教师及其他人员)表示对自己的谢意,这不仅是一种礼貌,也是对他人劳动的尊重,是治学者应当遵循的学术规范。内容限一页。
% modifier: 黄俊杰
% update date: 2017-04-15
%%

\chapter{致谢}

四年时光如白驹过隙,本科求学之路随之画上句点。回首往昔,我深感这段时光不仅丰富了我的知识与技能,提高了我的科学素养,更培养了我分析问题、解决问题的能力和不畏困难、深度求索的精神。
就读于中山大学的宝贵经历是我永久的财富。在此,我想衷心感谢那些给予我帮助与支持的老师、家人和同学们。

首先,我要特别感谢我的指导老师卢宇彤教授。作为一名本科生,我缺乏学术研究的经验,难以准确把握研究问题的要点,也难以评估自己的工作水平。而卢老师对所涉领域有着深刻的理解,她的广博知识和敏锐洞察力给予了我宝贵的指导。卢老师严谨的治学态度和勤奋的工作精神更是我学习的楷模。在此,我向卢老师致以崇高的敬意和诚挚的感谢。

其次,我要感谢我的家人。在我面临困难和挑战时,他们给予了我无私的支持与鼓励,让我坚持不懈、保持信心。正是因为他们的支持和付出,我才能够坚定不移地追求目标,在求学进步的道路上笃定前进。

最后,我要感谢所有曾经帮助过我的朋友们。你们的热心支持与耐心鼓励永远是我前行路上最坚实的后盾,我会倍加珍惜这份情谊,不忘初心,继续前行。

\vskip 108pt
\begin{flushright}
	卢科州\makebox[1cm]{} \\
	\today
\end{flushright}

