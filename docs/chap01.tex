%%
% 引言或背景
% 引言是论文正文的开端,应包括毕业论文选题的背景、目的和意义;对国内外研究现状和相关领域中已有的研究成果的简要评述;介绍本项研究工作研究设想、研究方法或实验设计、理论依据或实验基础;涉及范围和预期结果等。要求言简意赅,注意不要与摘要雷同或成为摘要的注解。
% modifier: 黄俊杰(huangjj27, 349373001dc@gmail.com)
% update date: 2017-04-15
%%

\chapter{绪论}
%定义,过去的研究和现在的研究,意义,与图像分割的不同,going deeper
\label{cha:introduction}
\section{选题背景与意义}
\label{sec:background}
% What is the problem
% why is it interesting and important
% Why is it hards, why do naive approaches fails
% why hasn't it been solved before
% what are the key components of my approach and results, also include any specific limitations,do not repeat the abstract
%contribution
量化交易是指利用数学模型和计算机算法辅助交易者分析金融市场行情,进行决策与交易的方法。
因子作为表征市场行情的关键数据,通常直接指示交易策略的执行。
早期因子以基于行情数据的经验公式为主,近年来伴随机器学习技术的发展,表征和抗噪能力优越的因子模型被交易者广泛使用。
其中,多分支因子模型通过多分支综合处理特征来分析多种行情指标,具有较为优良的行情表征能力。

在量化交易中,高频交易往往需要低延时交易系统自动完成行情的分析并基于分析执行策略。
交易系统的执行过程包括网关接收和解析交易所行情数据,预处理行情数据获取统计特征,特征输入因子模型推理计算,参考模型输出执行策略,通过网关完成下单或撤单操作等若干步骤。
高频交易对延时有极为苛刻的要求,其运行的关键路径应当在数十乃至数微秒内执行完成,从而能够正确执行策略,这对于交易系统的设计和实现提出了极高的要求。
在交易系统的关键路径中,因子模型的推理在全过程延时的占比较大,因此提高因子模型推理效率是顺利执行策略的关键。
不同于服务端高并发业务场景下的现有多数推理框架,低延时交易系统中使用的推理框架应当以单样本高频率计算密集场景下的低延时作为优化的最终目标。

通过充分利用因子模型的结构特征与输入模式,因子模型推理框架能够有效提高模型整体的计算效率,有效降低模型推理延时。
这将避免错失目标行情,从而改善交易系统的稳健性,显著提高交易策略的执行水平。

\section{国内外研究现状和相关工作}
\label{sec:related_work}

量化交易的发展历程中,因子模型经历了从传统单因子到多因子,再到机器学习驱动的优化过程。
早期量化投资依赖单一因子,如威廉·夏普提出的资本资产定价模型(CAPM)\cite{sharpe1964capital},通过市场风险因子(β)解释资产收益,奠定因子投资理论基础。
然而CAPM的局限性在于仅依赖单一市场因子,无法充分解释股票收益的异质性。
尤金·法马和肯尼斯·弗伦奇提出的三因子模型\cite{fama1993common}增加了规模因子(SMB)和价值因子(HML),由此进一步提出了五因子模型\cite{fama2015five},新增盈利能力因子和投资水平因子,显著提升了模型对股票收益的解释能力。
近年来,机器学习技术引入,因子模型构建和优化进入新阶段。
例如,基于XGBoost和SVM的多因子选股策略在A股市场表现出色,年化收益率显著高于传统模型\cite{bianchi2021bond}。
随机森林和XGBoost等算法能够更好地捕捉股票数据中的非线性关系\cite{gu2021autoencoder},显著提高模型的表征能力。
进一步,深度学习算法因其强大的特征提取和模式识别能力,被广泛应用于量化交易中\cite{chen2024deep}。
这些算法能够挖掘出更具预测性的因子组合\cite{kozak2020shrinking},从而显著提升模型的性能。
尤其是Transformer模型在处理时间序列数据方面表现出色,能够有效捕捉金融市场的动态变化和长期依赖关系,为高频交易策略提供了更精准的决策依据\cite{barez2023exploringadvantagestransformershighfrequency}。
多分支因子模型则实现回测质量和计算效率的合理平衡,被广泛应用到多品类交易的业务场景中。
量化交易的理论与实践不断迭代,为投资者提供了更强大的工具以应对复杂多变的市场环境。

尽管量化因子模型在近年来取得了显著进展,但其在高频交易场景下的推理速度仍面临挑战。
研究指出\cite{hansen2006realized},高频数据的噪声和复杂性使得传统统计方法在处理时效率低下,难以满足实时交易的需求。
此外,高频市场中的微观结构噪声和流动性波动进一步增加了模型的计算负担,限制了其在实际交易中的应用。
为了解决这一问题,一些学者提出了基于并行计算和分布式架构的优化方案。
例如,研究人员开发了一种基于GPU加速的高频交易算法\cite{frey2023jax},显著提升了模型的计算效率。
这也将因子模型推理加速方案的发展推进到针对具体业务场景综合多种因素共同优化的阶段。

在模型推理方面,当前多数模型推理框架针对服务端高并发场景进行优化。
例如,vLLM框架通过PagedAttention和Continuous Batching技术,在多GPU环境下实现了高吞吐量和优先级服务\cite{kwon2023efficientmemorymanagementlarge}。
Tencent TFCC通过使用MLIR(多级中间表示)技术实现了算子的自动融合,有效提高了访存效率并大幅节约了算子开发时间,从而显著降低了推理延时\cite{lattner2020mlircompilerinfrastructureend}。
TensorRT通过将模型的推理部署过程解耦为推理引擎构建和实例化部署两部分,从而能够在多个方面实现最优计算效率。
这些框架尽管对于高频交易的低延时要求缺乏针对性的解决方案,仍有较大的改进空间。
但同时也为高频交易场景下针对性优化的因子模型推理框架提供了先进的架构设计和技术应用范例。

在高频交易领域,除了算法和模型的优化,硬件加速技术也成为了研究的热点。
为了满足高频交易对速度的极致追求,研究人员开始探索利用专用硬件,如现场可编程门阵列(FPGA)和专用集成电路(ASIC),来加速因子模型的计算。
这些硬件设备具有低延迟、高吞吐量的特点,能够显著提升模型的推理速度,使其更适应高频交易的需求。
例如,FPGA可以通过定制化的电路设计,实现对特定计算任务的加速,减少模型推理的延迟,从而提高交易的执行效率\cite{ALI20241}。
然而硬件加速本身的成本过于高昂,交易者难以承担频繁的模型结构改动带来的损失。

总之,现有因子模型推理效率的提高需要推理框架本身具有优良的设计,并且针对具体的部署场景作针对性的优化。
对于多分支因子模型这一常用因子模型类型,推理框架在针对高频交易场景的低延时优化方面仍有较大的优化空间,需要进一步的研究和创新。

\section{本文的工作和贡献}

本文的主要贡献可以总结为以下几点:
\begin{enumerate}
\item 设计和实现了一个高可扩展性和过程解耦的模型推理框架
\item 提出了系统友好的内存管理方案和硬件友好的算子调优与动态绑定方案
\item 实现了基于多分支因子模型的拓扑分支分割算法和分支缓存机制
\end{enumerate}

\section{本文的论文结构与章节安排}

\label{sec:arrangement}

本文共分为五章,各章节内容安排如下:

第一章:绪论。本章阐释了选题背景与研究意义,对国内外相关研究进展作总结分析,并概述了本文的相关工作和贡献。

第二章:相关工程技术背景。本章阐述了交易实践中低延时系统的组成部分和功能,因子模型的设计原理和结构特征,介绍了现代推理框架的基本功能与结构,并针对选题适用的CPU高性能计算体系作概括阐述

第三章:框架和算法设计方法。系统介绍了本文所作推理框架的工程设计和实现方法,同时阐述了针对选题模型结构特性和输入特点实现的优化方案和创新算法。  

第四章:实验结果与分析。本章介绍了实验环境,通过对比多种开源推理框架,对本文实现的因子模型推理框架的性能进行测试,并对实验结果作总结概括。

第五章:总结与展望。概括总结了本文的工作和贡献,分析阐释了本文工作的局限,同时展望进一步的研究方向。
