%%
% 引言或背景
% 引言是论文正文的开端,应包括毕业论文选题的背景、目的和意义;对国内外研究现状和相关领域中已有的研究成果的简要评述;介绍本项研究工作研究设想、研究方法或实验设计、理论依据或实验基础;涉及范围和预期结果等。要求言简意赅,注意不要与摘要雷同或成为摘要的注解。
% modifier: 黄俊杰(huangjj27, 349373001dc@gmail.com)
% update date: 2017-04-15
%%

\chapter{绪论}
%定义,过去的研究和现在的研究,意义,与图像分割的不同,going deeper
\label{cha:introduction}
\section{选题背景与意义}
\label{sec:background}
% What is the problem
% why is it interesting and important
% Why is it hards, why do naive approaches fails
% why hasn't it been solved before
% what are the key components of my approach and results, also include any specific limitations,do not repeat the abstract
%contribution
量化交易是指利用数学模型和计算机算法辅助交易者分析金融市场行情,进行决策与交易的方法。
因子作为表征市场行情的关键数据,通常直接指示交易策略的执行。
早期因子以基于行情数据的经验公式为主,近年来伴随机器学习技术的发展,表征和抗噪能力优越的因子模型被交易者广泛使用。
其中,多分支因子模型通过多分支综合处理特征来分析多种行情指标,具有较为优良的行情表征能力。

在量化交易中,高频交易往往需要低延时交易系统自动完成行情的分析并基于分析执行策略。
交易系统的执行过程包括网关接收和解析交易所行情数据,预处理行情数据获取统计特征,特征输入因子模型推理计算,参考模型输出执行策略,通过网关完成下单或撤单操作等若干步骤。
高频交易对延时有极为苛刻的要求,其运行的关键路径应当在数十乃至数微秒内执行完成,从而能够正确执行策略,这对于交易系统的设计和实现提出了极高的要求。
在交易系统的关键路径中,因子模型的推理在全过程延时的占比较大,因此提高因子模型推理效率是顺利执行策略的关键。
不同于服务端高并发业务场景下的现有多数推理框架,低延时交易系统中使用的推理框架应当以单样本高频率计算密集场景下的低延时作为优化的最终目标。

通过充分利用因子模型的结构特征与输入模式,因子模型推理框架能够有效提高模型整体的计算效率,有效降低模型推理延时。
这将避免错失目标行情,从而改善交易系统的稳健性,显著提高交易策略的执行水平。

\section{国内外研究现状和相关工作}
\label{sec:related_work}

在量化交易发展过程中,因子模型经过了从单因子到多因子,再到机器学习因子模型的演变过程。
早期的量化投资依赖单一因子,如威廉·夏普提出的资本资产定价模型(CAPM)\cite{sharpe1964capital},其通过市场风险因子解释资本收益,奠定因子投资的理论基础。
而尤金·法马和肯尼斯·弗伦奇提出的三因子模型\cite{fama1993common}增加了规模因子和价值因子,由此进一步提出了新增盈利能力因子和投资水平因子的五因子模型\cite{fama2015five},从而显著提高因子模型对资本市场的表征水平。
随后,伴随着机器学习技术发展,因子模型的构建与优化进入了新的阶段。
基于SVM的多因子选股策略在A股市场表现出色,其年化收益率高于传统因子模型\cite{bianchi2021bond}。
而随机森林和XGBoost等算法能够更好地表征行情数据中的非线性关系\cite{gu2021autoencoder},从而显著提高因子模型的性能。
近年来,深度学习算法因其强大的特征提取和模式识别能力,被广泛应用于量化交易中\cite{chen2024deep}。
这些算法能有效挖掘出更具预测性的因子组合\cite{kozak2020shrinking},其对行情数据的表征能力显著提高。
尤其是Transformer模型在处理时间序列数据方面表现尤为突出,注意力机制能够有效捕捉行情数据中的长期依赖关系,从而为高频交易策略提供了更精准的决策依据\cite{barez2023exploringadvantagestransformershighfrequency}。
多分支因子模型则实现了表征性能与计算效率的合理平衡,因此在高频交易场景下得到广泛的应用。
因子模型伴随技术更新而不断迭代,为交易者提供了强大的工具以应对复杂多变的市场环境。

伴随因子模型在性能方面的提升,因子模型推理所需计算量也在迅速增加,使现有计算方法难以满足高频交易场景下的低延时需求。
Hansen指出\cite{hansen2006realized},面对高频交易中行情数据的复杂特征,传统特征计算方法的执行效率难以满足实时交易的需求。
而高频市场中的微观结构噪声进一步增加了因子模型的计算负担,限制了其在实际高频交易场景下的应用。
为解决这一问题,一些学者提出基于并行计算和分布式计算架构的模型推理优化方案。
例如,Frey\cite{frey2023jax}开发了一种基于GPU加速的高频交易算法,利用异构计算显著提升模型的计算效率。
这也将因子模型推理加速方案的研究转到针对具体业务场景综合多种因素共同优化的方向。

在模型推理方面,当前多数模型推理框架针对服务端高并发场景进行优化。
例如,vLLM框架通过PagedAttention和Continuous Batching技术,在多GPU环境下实现了高吞吐率和优先级服务\cite{kwon2023efficientmemorymanagementlarge}。
Tencent TFCC通过使用MLIR(多级中间表示)技术实现了算子的自动融合,有效提高了访存效率并大幅节约了算子开发时间,从而显著降低了推理延时\cite{lattner2020mlircompilerinfrastructureend}。
TensorRT通过将模型的推理部署过程解耦为推理引擎构建和实例化部署两部分,从而能够在多个方面实现最优计算效率。
这些框架尽管对于高频交易的低延时要求缺乏针对性的解决方案,仍有较大的改进空间。
但同时也为高频交易场景下针对性优化的因子模型推理框架提供了先进的架构设计和技术应用范例。

而除了算法和模型方面的优化,为高频交易场景的特殊需求设计制造专用硬件也成为提高模型推理效率的热门研究方向。
为满足高频交易对低延时的需求,现场可编程门阵列和专用集成电路被应用到因子模型推理模块以显著加速因子模型的计算。
这些硬件设备基于低延迟需求进行针对特定计算过程优化设计,不仅显著降低因子模型推理延时,而且在系统功耗和散热等方面具有优势。
例如,FPGA通过定制化的电路设计,实现对矩阵乘法的加速,显著减小模型线性层推理的延迟,以提高交易系统整体的执行效率\cite{ALI20241}。
但另一方面硬件加速本身的设计和制造成本过于高昂,交易者难以承担频繁的模型结构改动带来的损失,标准化硬件仍然是多数交易者的选择。

总之,现有因子模型推理效率的提高需要推理框架本身具有优良的设计,并且针对具体的部署场景作针对性的优化。
对于多分支因子模型这一常用因子模型类型,推理框架在针对高频交易场景的低延时优化方面仍有较大的优化空间,需要进一步的研究和创新。

\section{本文的工作和贡献}

为了降低多分支因子模型在高频交易场景下的推理延时从而提高交易策略的执行水平,
本文通过结合高频交易场景的业务需求,同时基于多分支因子模型的输入模式与结构特征,
提出和实现了一个低延时的多分支因子模型推理框架。
本文的主要贡献可以总结为以下几点:
\begin{enumerate}
\item 框架将模型优化部署的过程解耦为静态预处理和动态运行两个阶段。其中静态预处理阶段能够在模型部署前通过优化模块对模型执行算子融合、模型剪枝等计算图优化,并通过结合CPU缓存和指令集架构优化算子性能。另一方面也为动态运行过程中的优化方法提供内存预分配、计算图重建等辅助。而动态运行部分则结合运行过程中的运行环境和模型推理的中间数据,动态地执行如缓存预取等优化方法。通过静态预处理和动态运行两阶段优化,多分支因子模型推理框架具有了高可拓展性,拓展了因子模型优化空间。
\item 框架提出了系统友好的内存管理方案和硬件友好的算子调优与动态绑定方案。框架结合Linux系统的大页内存、内存映射和内存锁定等机制,消除了访存缺页中断导致的延时,并且通过使用内存池,减少运行过程中的内存分配与回收,提高计算效率。同时框架通过Blaze HPC,在静态预处理阶段综合CPU缓存和指令集架构信息进行算子编译调优。在动态运行期间,则基于输入张量形状和数据类型等信息实现更高效算子的动态绑定。
\item 框架在静态预处理阶段设计实现了计算图的拓扑分支分割算法,并在动态运行阶段实现了基于分支结构的缓冲机制。框架通过位运算为切片节点赋值唯一分支标识,然后从所有切片节点遍历计算图为所有节点通过或运算标识其所属分支,从而完成计算图的分支分割,随后基于分支完成计算图重建。在动态运行阶段,模型转而以分支为单位执行推理,通过匹配上一个输入和预计算哈希表查询,实现缓存命中时直接返回缓存结果,从而大幅减小计算量降低推理延时。
\end{enumerate}

\section{本文的论文结构与章节安排}

\label{sec:arrangement}

本文共分为五章,各章节内容安排如下:

第一章:绪论。本章阐释了选题背景与研究意义,对国内外相关研究进展作总结分析,并概述了本文的相关工作和贡献。

第二章:相关工程技术背景。本章阐述了交易实践中低延时系统的组成部分和功能,因子模型的设计原理和结构特征,介绍了现代推理框架的基本功能与结构,并针对选题适用的CPU高性能计算体系作概括阐述

第三章:框架和算法设计方法。系统介绍了本文所作推理框架的工程设计和实现方法,同时阐述了针对选题模型结构特性和输入特点实现的优化方案和创新算法。  

第四章:实验结果与分析。本章介绍了实验环境,通过对比多种开源推理框架,对本文实现的因子模型推理框架的性能进行测试,并对实验结果作总结概括。

第五章:总结与展望。概括总结了本文的工作和贡献,分析阐释了本文工作的局限,同时展望进一步的研究方向。
