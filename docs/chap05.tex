%% chapter 5 dataset, network structure, experiment and result
\chapter{总结与展望}
\label{cha:experiment}

\section{本文工作总结}
本文对高频交易场景的业务特征和多分支因子模型的输入模式与结构特征进行分析,在此基础上设计和实现了一个具有高可拓展性与过程解耦合的因子模型推理框架。
在这一推理框架下,我们设计了更详细的加速方案:
首先是利用系统机制有效消除缺页中断和提高访存速度的内存管理方法;
其次设计了针对CPU架构完成算子调优的工作流,并实现了动态算子绑定的拓展机制;
最后,框架充分利用了分支输入的可重复性,实现了基于多分支因子模型的分支划分算法和缓存机制。
实验证明,对于多数输入特征,通过选取恰当的缓存方法,框架相较于当前推理框架具有优越的推理性能,实现了从 1.08 倍到 3.82 倍的推理性能提升。


\section{讨论和展望}
尽管本文提出的框架方案相较已有相关工作显著降低了延时,但仍存在许多需要改进的方面。在本小节中,我们将概述本文工作的不足与局限,并展望未来的研究方向。

首先,针对多分支因子模型的结构特征,仍然可能存在进一步的优化空间,多分支结构的分支之间互相独立,因此各分支的推理过程存在线程并行的加速空间。
同时考虑到并行线程面向不同CPU架构时存在核心绑定相关调优问题,因此亟待进一步开发与研究。

其次,系统友好的内存管理方案和硬件友好的算子调优和动态绑定方案仍然具有优化空间。
当前推理进程仍然运行在操作系统用户态,仍然存在大量无关进程影响交易系统运行,
理想情况下,应当使整个交易链路直接运行在内核态从而最大限度减少由于系统调用中断导致的延时。

最后,尽管本文提出和实现了一个过程解耦和具有高可拓展性的因子模型推理框架,但是当前的研究仅仅局限于针对多分支因子模型的特性设计和实现优化算法。
伴随着量化交易的进一步发展,当前的优化方法很可能不再适用于具有更优性能的不同结构因子模型,本文提出的方案仍然需要进一步的开发研究。
